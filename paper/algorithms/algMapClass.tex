\begin{figure}[t!]
        \begin{algorithmic}[1]
                \Function{cluster\_matching}{{\tt dates}}\\
                \COMMENT{\textbf{Input:} ~~~{\tt dates} to process. Each date has} 
                \Statex \hspace{1.6cm} an associated list of physical models and a 
                \Statex \hspace{1.6cm} a reference (segmentation) map.\\
                \COMMENT{\textbf{Output:}{\tt ~model\_maps} of the new, missing, and}
                \Statex \hspace{1.6cm} matched coronal holes for each physical model
                \Statex \hspace{1.6cm} and date are stored as separate files.
                \Statex
                \State~\C{Process each date separately}
                \For{ {\tt date} $\in$ {\tt dates}}
                \State \C{Read  and process reference image}
                \State {\tt ref\_map} $\gets$ {\bf load\_ref\_data}({\tt date})
                \State {\tt ref\_map$_{\{+,-\}}$} $\gets$ {\bf pre\_process}({\tt ref\_map})\label{alg:ov_preprocessing1}
                \Statex
                \State~\C{Process associated physical models}
                \For{ {\tt model} $\in \{{\tt model\_1}, \dots, {\tt model\_M}\}$ }
                \State {\tt model\_map} $\gets$ {\bf load\_model}({\tt date}, {\tt model})
                \State {\tt model\_map$_{\{+,-\}}$}  
                \Statex \hspace{3.7cm}$\gets$ {\bf pre\_process}({\tt model\_map})\label{alg:ov_preprocessing2}
                \Statex
                \State~\C{Analyze each polarity separately}
                \For{polarity {\tt p} $\in$ \{ $+$, $-$\}}
                \State {\textbf{\textit{Cluster}}} coronal holes that are are close.
                \State {\textbf{\textit{Detect}}}  coronal hole clusters that are in
                \Statex \hspace{2.9cm} physical maps but not in reference
                \Statex \hspace{2.9cm} map using Mahalanobis distance
                \Statex \hspace{2.9cm}  and store the results in
                \Statex \hspace{2.9cm} {\tt new\_map$_p$} and {\tt missing\_map$_p$}
                \State {\textbf{\textit{Re-cluster}}} remaining coronal holes in 
                \Statex \hspace{2.9cm} {\tt ref\_map} and {\tt model\_map} to
                \Statex \hspace{2.9cm} achieve equal number of clusters.
                \State {\textbf{\textit{Match}}} clusters using \textit{linear programming} 
                \Statex \hspace{2.9cm} and save the results in
                \Statex \hspace{2.9cm} {\tt matched\_map$_p$}                                                            
                \EndFor
                %\State {\small \textbf{\textit{Extract}}} {\tt features}  from {\tt new\_map$_p$}, 
                %\Statex \hspace{2.53cm} {\tt missing\_map$_p$} and {\tt missing\_map$_p$}
                %\Statex \hspace{2.53cm} for polarity {\tt p} $\in$ \{+, -\}.         
                %\State {\small \textbf{\textit{Classify}}} {\tt model} using extracted features.
                %\label{algLine:get_features}
                \Statex
                \State~\C{Save maps for each physical model}
                \State {\tt model\_maps} $\gets$  ({\tt new\_map$_p$}, 
                \Statex \hspace{2.2cm} {\tt missing\_map$_p$}, {\tt matched\_map$_p$})
                \Statex \hspace{2.2cm} for polarity {\tt p} $\in$ \{+, -\}.
                \State {\textbf{\textit{Store}}} {\tt model\_maps}
                \EndFor
                \EndFor     
                \EndFunction
        \end{algorithmic}
        \caption{Cluster matching algorithm. The algorithm process the segmentation map
                  (reference map) against the physical models for each given date. 
                  The resulting maps contain the new, missing, and matched coronal holes
                       between the reference map and each physical model.
                  In our example, we have 12 physical maps ($M=12$).     }
        \label{Fig:autoClassOverviewAlg}
      \end{figure} 
