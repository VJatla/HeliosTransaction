% [28] is the number of lines to occupy. This has to be done manually.
\begin{figure*}[!t]  
  \centering
  \resizebox{0.88\textwidth}{!}{%
    \begin{tikzpicture}[node distance=2cm]
	    \node(syn)[EMPTY, draw=none, text width=2.5cm, text centered, 
	    minimum width = 1.5cm, minimum height = 1.5cm, scale=1.3]
	    {
	    	Synoptic maps
	    };
	    \node(mag)[EMPTY, text width=2.5cm, text centered,below of =syn, node distance = 4cm,
	    rounded corners, minimum width = 1.5cm, minimum height = 1.5cm, scale=1.3]
	    {
	    	Magnetic maps
	    };
	    

	    \node(fcn)[EMPTY, rounded corners, draw=red!50, node distance = 6 cm, line width=0.5pt,right of= syn, yshift = 15mm, minimum width = 1.5cm, minimum height = 1.5cm, scale=1.3,yshift=-12mm]
	    {
	    	\small{FCN}
	    };
    	    \node(hh)[EMPTY,rounded corners, draw=red!50,   node distance = 4 cm, line width=0.5pt,below of= fcn, minimum width = 1.5cm, minimum height = 1.5cm, scale=1.3]
	    {
	    	\small{Henney\\Harvey}
	    };
        	    \node(comb)[EMPTY,rounded corners, draw=red!50, node distance = 4 cm, line width=0.5pt,right of= hh,yshift=20mm, minimum width = 1.5cm, minimum height = 1.5cm, scale=1.3]
    {
    	\small{Combine}
    };

     \node(ls)[EMPTY,rounded corners, draw=red!50, node distance = 4.5 cm, line width=0.5pt,right of= comb, minimum width = 1.5cm, minimum height = 1.5cm, scale=1.3]
	{
		\small{Level sets}
	};
	    \node [ BOXL, minimum width = 13cm, minimum height = 7cm, line width = 2pt, fit={(fcn) (hh) (comb) (ls)}](segbox) {
	    	\vspace{5cm}
	    	\Large{Segmentation}
    	};
	    
	    
      \node(cm)[BOXL, 
right of = hh, node distance = 8cm,line width=2pt,
minimum width = 1.5cm, minimum height = 1.5cm, scale=1.3,xshift=50mm
]
{
	Clustering \\and\\ Matching 
};

      % Classification
\node(class)[BOXL, right of = cm, line width = 2pt, node distance = 5.5cm,
minimum width = 1.5cm, minimum height = 1.5cm, scale=1.3]
{
	Classification
};



      \node (obs) [BOXL, line width=2pt,
minimum width = 1.5cm, minimum height = 1.5cm, scale=1.3, above of=class, yshift=8mm]
{
	Solar model\\
	generation
	% Solar Model \\ Simulation
};
      \node (sim)[BOXL, left of=obs, node distance = 5.5cm,line width=2pt,
minimum width = 1.5cm, minimum height = 1.5cm, scale=1.3]
{
	Coronal hole\\
	maps\\extraction
	% Pred. Coronal Hole Maps.
};


% Input arrows
\draw[ARROW, line width=2pt] (syn.east) to [|-] (fcn.west);
\draw[line width=2pt] (syn.east) to [|-] (segbox.west);
\draw[ARROW,line width=2pt] (segbox.west) -| (hh.north);
\draw[ARROW, line width=2pt] (mag.east) to [|-] (hh.west);
\draw[line width=2pt] (mag.east) to [|-] (segbox.207);
\draw[ARROW,line width=2pt] (segbox.207) -| (ls.south);





\draw[ARROW, line width=2pt] (fcn.east) -| (comb.north);
\draw[ARROW, line width=2pt] (hh.east) -| (comb.south);
\draw[ARROW, line width=2pt] (comb.east) to[-|-] (ls.west);
\draw[line width=2pt] (ls.east) to[-|-] (segbox.east);
% Segmentation to Matching
\draw[ARROW, line width=2pt] (segbox.east) to[-|-] (cm.west);
\draw[ARROW, line width=2pt] (cm.east) to[-|-] (class.west);
% Model generation
\draw[ARROW, line width=2pt] (obs.west) to[-|-] (sim.east);
\draw[ARROW, line width=2pt] (sim.south) to (cm.north);

    \end{tikzpicture}
  }
  \caption{\label{fig:IntroBlkDiag}
  	\color{blue}
    General system diagram.
    A collection of coronal hole maps are generated for
    different physical model parameters.
    The input observations are used to generate a candidate
    coronal hole map.
    A hierarichical clustering and matching algorithm
    is used for matching clusters of coronal holes.
    The final classifier is based on the matching results.
  }
\end{figure*}
