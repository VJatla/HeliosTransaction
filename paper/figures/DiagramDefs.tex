% This file contains all of the definitions for
% use with the diagrams.


% Use the following rules for MIXED horizontal/vertical paths.
% from: http://tex.stackexchange.com/questions/45347/vertical-and-horizontal-lines-in-pgf-tikz               
% Modified by adding the two top styles!
% Example use:  \draw[->,red] (2.1,1.1) to[|-|=.2] (1.1,0.1);
\tikzset{
	%
	% Added: Right angle connection.
	%
	|-/.style={   
		to path={ (\tikztostart) |- (\tikztotarget) \tikztonodes }
	},
	%
	% Horizontal -> vertical -> horizontal line
	%
	-|-/.style={  % .
		to path={
			(\tikztostart) -| ($(\tikztostart)!#1!(\tikztotarget)$) |- (\tikztotarget)
			\tikztonodes
		}
	},
	-|-/.default=0.5, % default at mid-point.
	%
	% Vertical -> horizontal -> horizontal line.
	%
	|-|/.style={   
		to path={
			(\tikztostart) |- ($(\tikztostart)!#1!(\tikztotarget)$) -| (\tikztotarget)
			\tikztonodes
		}
	},
	|-|/.default=0.5,  % This is the mid-point default. It is argument #1.
}


% Default text=left aligned.
% Define styles for the different diagram components:

% Database style. Produces a video database for connections.
\tikzstyle{DB}     = [cylinder, line width=0.5mm, text width=2cm, text centered, 
                      %cylinder uses custom fill, cylinder body fill=yellow!50, cylinder end fill=yellow!50, 
                      shape border rotate=90, aspect=0.25, draw]

% Empty shape. Use this for passing arguments to the code.
\tikzstyle{EMPTY} = [draw=none, text width=2.5cm, text centered, font=\fontsize{15}{0}\selectfont] 

% Red box. This is the default box processing.
\tikzstyle{BOX}   = [draw=red,         % color for LINES
                     ultra thick,   % control line width.
                     %fill=blue!50,    % Fill-in the box.
                     text centered,
                     rectangle,        % Box shape
                     text width=2.5cm] % Controls how much text we are putting inside. 

% Red box. This is the default box processing.
\tikzstyle{BOXL}   = [draw=red,         % color for LINES
                     line width=3pt,   % control line width.
                     %fill=blue!50,    % Fill-in the box.
                     text centered,
                     rectangle,        % Box shape
                     text width=2.5cm] % Controls how much text we are putting inside. 
                     
\tikzstyle{BOXL2}   = [	draw=red,         % color for LINES
						line width=2pt,   % control line width.
						%fill=blue!50,    % Fill-in the box.
						text centered,
						rectangle,        % Box shape
						text width=3cm,
						font=\fontsize{15}{0}\selectfont] % Controls how much text we are putting inside. 
						
\tikzstyle{BOXL3}   = [	draw=red,         % color for LINES
line width=1pt,   % control line width.
%fill=blue!50,    % Fill-in the box.
text centered,
rectangle,        % Box shape
text width=3cm,
font=\fontsize{15}{0}\selectfont] % Controls how much text we are putting inside. 

                     
\tikzstyle{BOXLD}   = [	draw=red,         % color for LINES
						line width=3pt,   % control line width.
						dashed,			  % Dashed box
						%fill=blue!50,    % Fill-in the box.
						text centered,
						rectangle,        % Box shape
						text width=2.5cm] % Controls how much text we are putting inside.
						
\tikzstyle{BOXLD2}   = [	draw=red,         % color for LINES
line width=2pt,   % control line width.
dashed,			  % Dashed box
%fill=blue!50,    % Fill-in the box.
text centered,
rectangle,        % Box shape
text width=2.5cm,
font=\fontsize{30}{0}\selectfont] % Controls how much text we are putting inside. 


 \tikzstyle{BOXB}   = [draw=black,         % color for LINES  --> Added by vj
					 line width=3pt,   % control line width.
					 %fill=blue!50,    % Fill-in the box.
					 text centered,
					 rectangle,        % Box shape
					 text width=2.5cm] % Controls how much text we are putting inside. 

% Red smaller box. Use this to fit more boxes.
\tikzstyle{BOXs} = [draw=red,         % color for LINES
                     ultra thick,   % control line width.
                     %fill=blue!50,    % Fill-in the box.
                     text centered,
                     rectangle,        % Box shape
                     text width=2cm] % Controls how much text we are putting inside.

% Communications box. Use for exporting data.
\tikzstyle{BOXC}  = [draw=red,       % color for LINES
                     ultra thick, % control line width.
                     text centered,  
                     %fill=blue!50,   % Fill-in the box.
                     rectangle,      % Box shape
                     text width=3cm] % Controls how much text we are putting inside. 

% Background styles for communications:
\tikzstyle{queryComm}   = [fill=yellow!20,rounded corners, draw=black!50, dashed]
\tikzstyle{designComm} = [fill=blue!20,rounded corners, draw=black!50, dashed]

% ARROWs should never use hard-coded dimensions! See the top for connections.  
\tikzstyle{ARROW}  = [ultra thick, ->, >=stealth]             % regular black arrow
\tikzstyle{DARROW} = [dashed, ultra thick, ->, >=stealth]     % dashed arrow for communications
\tikzstyle{IARROW} = [draw=blue, fill=blue, ultra thick, ->]  % input arrow (blue)
\tikzstyle{OARROW} = [draw=red, ultra thick, ->] % output arrow (green)


% Small version of the BOX. It supports smaller boxes.
\tikzstyle{SBOX}   = [draw=blue, % color for LINES
              ultra thick,       % control line width.
              %fill=blue!50,     % Fill-in the box.
              text centered,
              rectangle,         % Box shape
              text width=1.6cm]  % Controls how much text we are putting inside. 
