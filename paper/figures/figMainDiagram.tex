% [28] is the number of lines to occupy. This has to be done manually.
\begin{figure*}[!t]  
  \centering
  \resizebox{0.88\textwidth}{!}{%
    \begin{tikzpicture}[node distance=2cm]
      
      % Solar Model simulation
      \node (obs) [BOXL, line width=2pt,
      minimum width = 1.5cm, minimum height = 1.5cm, scale=1.3]
      {
        Solar model\\
        generation
        % Solar Model \\ Simulation
      };
      
      \node (sim)[BOXL, right of = obs, node distance = 5cm,line width=2pt,
      minimum width = 1.5cm, minimum height = 1.5cm, scale=1.3]
      {
        Coronal hole\\
        maps\\extraction
        % Pred. Coronal Hole Maps.
      };
      
      % Input
      \node(syn)[EMPTY, draw=none, text width=2.5cm, text centered,below of =obs, node distance = 3.5cm,
       minimum width = 1.5cm, minimum height = 1.5cm, scale=1.3]
      {
        Synoptic maps
      };
      \node(mag)[EMPTY, text width=2.5cm, text centered,below of =syn, node distance = 1cm,
      rounded corners, minimum width = 1.5cm, minimum height = 1.5cm, scale=1.3]
      {
        Magnetic maps
      };
      \node(seg)[BOXL, node distance = 4 cm, line width=2pt,below of =sim, minimum width = 1.5cm, minimum height = 1.5cm, scale=1.3]
      {
        Coronal hole \\Segmentation
      };

      % Matching
      \node(bigBox)[BOXL, 
      right of = seg, node distance = 6.5cm,line width=2pt,
      minimum width = 3.2cm, minimum height = 1.5cm, scale=1.3
      ]
      {
        Clustering and Matching 
      };


      % Classification
      \node(class)[BOXL, right of = bigBox, line width = 2pt, node distance = 5.5cm,
      minimum width = 1.5cm, minimum height = 1.5cm, scale=1.3]
      {
        Classification
      };

      % Connecting boxes
      % \draw [ARROW, line width=2pt] (sim.east) to (pred.west);
      % \draw [ARROW, line width=2pt] (predicted.east)   to (match.west);
      % \draw [ARROW, line width=2pt] (pred.east)       to[-|-] (match.172      );
      % \draw [ARROW, line width=2pt] (match.east) to(class.west);
      % \draw [DARROW, line width=2pt] (class.east) to(output.west);
      \draw[ARROW, line width=3pt] (syn.east) to[-|-] (seg.172);
      \draw[ARROW, line width=3pt] (mag.east) to[-|-] (seg.188);
      \draw[ARROW, line width=3pt] (obs.east) to (sim.west);
      \draw[ARROW, line width=3pt] (seg.east) to  (bigBox.west);
      \draw[ARROW, line width=3pt] (bigBox.east) to (class.west);
      \draw[ARROW, line width=3pt] (sim.east) to[-|-] (bigBox.172);            
    \end{tikzpicture}
  }
  \caption{\label{fig:IntroBlkDiag}
    General system diagram.
    A collection of coronal hole maps are generated for
    different physical model parameters.
    The input observations are used to generate a candidate
    coronal hole map.
    A hierarichical clustering and matching algorithm
    is used for matching clusters of coronal holes.
    The final classifier is based on the matching results.
  }
\end{figure*}
